\documentclass{amsart}
\usepackage{fullpage}

\usepackage[T1]{fontenc}
\usepackage[utf8]{inputenc} 
\usepackage{lmodern}
\usepackage[slovene]{babel}
\usepackage{hyperref}
\usepackage{amsmath,amssymb,amsfonts, mathtools}
\usepackage{bbm}

\linespread{1.2}

\title{Descrete-p-dispersion problem}
\author{Tilen Klinc}
\author{Janez Podlogar}
\date{\today}

\begin{document}
    
\maketitle

\section{Opis problema}

Obravnavamo problem izbire $ p $ točk iz nabora $ n $ točk v nekem prostoru
tako, da je minimalna razdalja med izbranimi točkami maksimizirana. Cilje je
izbrati točke tako, da so čim bolj "razpršene". Problem se pojavi v kontekstu
logistike, ko je bližina poslopji nezaželjena.


Primer je izbira lokacij raketnih silosov. Če so silosi skoncentirani na enem
obmčju, jih lahko nasportnik uniči z enim napadom. Dlje kot so silosi med sabo,
manjše so možnosti, da bo nasportnih uničil vse z enim napadom. "Dobra" lokacija
silosov torej odvira od napada, saj bo v primeru, ko z enim napadom ne more uničiti
vseh silosov, deležen protinapada. V tem kontekstu je problem posplošitev reka 
"ne nosi vseh jajc v eni košari".

Drugi primer je lokacija franšižnih restavracij v mestu. Odločevalci imajo na voljo $ n $
lokacij za najem, želijo pa si jih odpreti $ p $. Ne želijo si odpreti dveh restavracij
v bližini, saj bi si delili isti bazen strank. Skupna prodaja bo višja, če so restavracije
"dobro razpršene" po mestu.

Omenimo, da bi v primeru raketnih silosov pride v poštev evklidska razdalja, v primeru
restavracij pa razdalja po mestnih ulicah. A na naše reševanje problema metrika ne bo 
vplivala.

\end{document}