\documentclass{article}
\usepackage{fullpage}

\usepackage[T1]{fontenc}
\usepackage[utf8]{inputenc} 
\usepackage{lmodern}
\usepackage[slovene]{babel}
\usepackage{hyperref}
\usepackage{amsmath,amssymb,amsfonts, mathtools}
\usepackage{bbm}
\usepackage{hyperref}
\usepackage{makeidx}

\makeindex

\linespread{1.1}

\title{Descrete $p$-dispersion problem}
\author{Tilen Klinc, Janez Podlogar}
\date{\today}

\begin{document}
    
\maketitle

\section{Opis problema}

Obravnavamo problem izbire $ p $ točk iz nabora $ n $ točk v nekem prostoru
tako, da je minimalna razdalja med izbranimi točkami maksimizirana. Cilje je
izbrati točke tako, da so čim bolj ``razpršene''. Problem se pojavi v kontekstu
logistike, ko je bližina poslopji nezaželjena.


Primer je izbira lokacij raketnih silosov. Če so silosi skoncentirani na enem
obmčju, jih lahko nasportnik uniči z enim napadom. Dlje kot so silosi med sabo,
manjše so možnosti, da bo nasportnih uničil vse z enim napadom. "Dobra" lokacija
silosov torej odvira od napada, saj bo v primeru, ko z enim napadom ne more uničiti
vseh silosov, deležen protinapada. V tem kontekstu je problem posplošitev reka 
``ne nosi vseh jajc v eni košari''.

Drugi primer je lokacija franšižnih restavracij v mestu. Odločevalci imajo na voljo $ n $
lokacij za najem, želijo pa si jih odpreti $ p $. Ne želijo si odpreti dveh restavracij
v bližini, saj bi si delili isti bazen strank. Skupna prodaja bo višja, če so restavracije
``dobro razpršene'' po mestu.

Omenimo, da bi v primeru raketnih silosov pride v poštev evklidska razdalja, v primeru
restavracij pa razdalja po mestnih ulicah. A na naše reševanje problema metrika ne bo 
vplivala.

\section{Formalizacija problema}

Za lokacije imamo podano lokacije $ I = \{ 1, 2, \ldots, n \} $ in matriko $ D \in 
\mathbb{R}^{n \times n} $, kjer nam $ d_{ij} \in D $ pove rezdaljo med lokacijo $ i $
in lokacijo $ j $. Brez škode za splošnost predpostavimo, da je matrika D simetrična in
da so njeni ne diagonalni elementi strogo pozitivni. Rešujemo optimizacijski problem
\begin{align*}
    & \max f(U) \\
    & \text{p.p. } f(U) = \min \{ d_{ij} \mid i,j \in I \text{ in } i \neq j \} \\
    & \qquad U \subseteq I \\
    & \qquad |U| = p .
\end{align*}

\section{Načrt dela}

Predstavila bova ``Kuby''\cite{kuby1987programming} formulacijo optimizacijskega problema 
in kompaktno linearno formulacijo problema opisano v ``Sayah in Irnich''\cite{sayah2017new}.

Problem bomo reševali minimizacijo v dani smeri oziroma \textit{line search} in z 
zgoraj omenjenim kompaktnim linearnim programom v katerega bomo vključili spodnje in zgornje omejitve
problema \cite{sayah2017new}. Oba algoritma bomo primerjali na naključno generiranih
točkah v ravnini.

Algoritme bova implementirana v programskem okolju \textbf{Sage}. V primeru, da bo
\textbf{Sage} pretežak se bova zanesla na \textbf{R} in knjižnico \textbf{lpsolve}
ali pa na \textbf{Python} in paket \textbf{SciPy}.





\bibliographystyle{plain}
\bibliography{literatura}
\printindex

\end{document}